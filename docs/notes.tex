\documentclass[10pt]{article}         %% What type of document you're writing.

%%%%% Preamble

%% Packages to use

\usepackage{amsmath,amsfonts,amssymb}   %% AMS mathematics macros
\usepackage[utf8]{inputenc}
\usepackage[english]{babel}
\usepackage{comment}

%% Title Information.

%\title{wavefronts dash the graph}
\title{Wavefront sequence-to-graph alignment}
\author{Andrea Guarracino and Erik Garrison}
%% \date{21 July 2222}           %% By default, LaTeX uses the current date

%%%%% The Document

\begin{document}

\maketitle

\begin{abstract}
  We explore an extension to the wavefront algorithm that generalizes it to work on sequence graphs.
  This follows the basic formulation of partial order alignment, wherein the recurrence relations defining alignment are extended to consider the topology of a target graph.
%  The use of different wavefront extension patterns allows us to apply the approach to cyclic graphs.
\end{abstract}


\section{Segmented sequence graph}

As input we take an alphabet $\Sigma$, a query sequence $q = \Sigma^n$, and a target sequence graph $G = (V, E \subseteq (V \times V ), \sigma : V \rightarrow \Sigma)$, where $V$ is the node set, $E$ is a set of directed edges, and $\sigma$ is a function that appoints one character to each node as the node label.
We denote the in-neighbors of a node as $\delta_v^{in} = \{x : (x, v) \in E\}$, and the out-neighbors as $\delta_v^{out} = \{x : (v, x) \in E\}$.
The segments of the graph $C$ are linear unbranching components made of series of nodes, $C = \{c = v_j \ldots v_{|j+c|} : \forall_{v_i, v_{i+1} \in c} (v_i, v_{i+1}) \in E \land |\delta_{v_i}^{out}| = 1 \land |\delta_{v_{i+1}}^{in}| = 1 \}$.
Segments correspond to compressible series of nodes that could be combined into a single node with a string label without disrupting the language of strings represented by the graph.
A graph $G$ with length $m = |V|$ can also be represented as a target sequence $T = \Sigma^m = t_0\ldots t_{m-1}$, where each character corresponds to a node in our graph, $\sigma(v_i) = t_i$.

\section{Sequence-to-graph alignment}

%In the process it should be created a WFA-DP-like table representing the possible alignments between the query and the targe.t
%\bigbreak

%Given a sequence graph $G = (V, E, \sigma)$, equivalently represented as a target sequence $t$ where each character corresponds to a node in our graph,
%and a query sequence $q = \Sigma^n$, the algorithm fills the WFA-DP-like table $M$ of size $m \times |V|$, where rows are indexed by integers representing base pairs in the sequence, and columns are indexed by nodes in the graph.
We define the pairwise global alignment between the query sequence $Q = q_0\ldots q_{n-1}$ and a graph $G \implies T = t_0\ldots t_{m-1}$ as the computation of the path from $(0, 0)$ to $(n, m)$ with minimum score, allowing for matches, mismatches, and gaps, each with costs defined by a different score function.
A trivial graph with a single segment $|C| = 1$ is equivalently represented as a sequence $T$.
In this case, we could apply standard pairwise alignment method like Smith-Waterman-Gotoh (SWG) \cite{smith1981comparison,gotoh1982improved} or the WaveFront Algorithm (WFA) \cite{Marco_Sola_2020} to derive an optimal alignment.
To formulate sequence-to-graph alignment, we first define algorithms for pairwise sequence alignment, and then show how these are generalized to the case where the target graph is not a single segment.

\section{Smith-Waterman-Gotoh}

In SWG, we maintain three matrices, $H$, $E$, and $F$.\footnote{To avoid ambiguity about the polarization of insertions and deletions, we adopt the notation of \cite{Farrar_2006} to formulate SWG and refer to indels specifically as gaps versus the query or target sequence.}
The value of a cell in $H_{i, j}$ represents the score of an alignment ending at the corresponding location in the query $q_i$ and the graph $t_j$, while $E$ contains scores ending with a gap extending along the target $T$ (or deletions in $Q$ relative to $T$) and $F$ contains scores ending in a gap extending along the query $Q$ (or insertions in $Q$ relative to $T$).
Our gap-affine penalty scores $p = {a, x, o, e}$ define the penalty for matching ($a$) or mismatching ($x$) any base of the target and query, while the gap-penalty function for a gap of $n$ bp is expressed as $g(n) = o + n \cdot e$.
A function $s(i, j)$ scores the bases in our query and target, yielding $a$ if $q_i = t_j$ and $x$ otherwise.
Equation \ref{eq:swg} shows the recurrence relations in standard SWG\@.

\begin{equation}
  \left\{\begin{array}{l}
  E_{i, j}=\min \left\{H_{i, j-1}+o+e, E_{i, j-1}+e\right\} \\
  F_{i, j}=\min \left\{H_{i-1, j}+o+e, F_{i-1, j}+e\right\} \\
  H_{i, j}=\min \left\{E_{i, j}, F_{i, j}, H_{i-1, j-1}+s\left(q_{i-1}, t_{j-1}\right)\right\}
  \end{array}\right.
  \label{eq:swg}
\end{equation}

\section{Partial order alignment}

We now extend these recurrence relations to consider the topology of the graph.
%we must extend these recurrence relations in SWG to consider $\delta_v^{in}$ for each node in $G$.
Insertions in the query relative to the target only require that we consider scores in the same node of the graph, so the relation defining $F$ remains unchanged.
However, both $H$ and $E$ must consider the set of inbound nodes $\delta_v^{in}$.
Assuming that we index nodes $v \in V$ in the same order as characters in $t$, then $\sigma(v_j) = t[j]$ and $\delta_j^{in}$ yields the set of indexes of inbound nodes of $v_j$ in $t$.
Equation \ref{eq:poa} shows the generalized recurrence relation in partial order alignment \cite{Lee_2002}, or SWG-POA\@.

\begin{equation}
  \left\{\begin{array}{l}
  E_{i, j}=\min_{\forall u \in \delta_j^{in}} \left\{H_{i, u}+o+e, E_{i, u}+e\right\} \\
  F_{i, j}=\min \left\{H_{i-1, j}+o+e, F_{i-1, j}+e\right\} \\
  H_{i, j}=\min \left\{E_{i, j}, F_{i, j}, \min_{\forall u \in \delta_j^{in}} \{ H_{i-1, u}+s\left(q_{i-1}, t_{u}\right)\}\right\}
  \end{array}\right.
  \label{eq:poa}
\end{equation}

\section{Wavefront algorithm}

WFA reformulates the SWG model to support the incomplete evaluation of the $H$ matrix while maintaining the guarantee that we find the lowest-cost path from $(0,0)$ to $(n,m)$.
Instead of filling the matrix uniformly, such as vertically or horizontally, WFA derives a series of wavefronts, each of which contains the set of cells in $H$ reached from $(0,0)$ with the same score.
Setting the match score $a = 0$ allows these wavefronts to extend in one step through consecutive matches.

For a score $s$ and diagonal $k = j - i$, the furthest-reaching point $\mathcal{R}_{s,k}$ indicates the cell in $H$ that is the furthest from the beginning of diagonal $k$ with score $s$.
$\widetilde{E}_{s, k}$, $\widetilde{F}_{s, k}$, and $\widetilde{H}_{s, k}$ store the offset in the diagonal to furthest-reaching point $\mathcal{R}_{s,k}$ in each of the SWG matrices.
For a given score $s$, the $s$-wavefront $\mathcal{W}_s$ is the set of offsets $\widetilde{E}_{s, k}$, $\widetilde{F}_{s, k}$, and $\widetilde{H}_{s, k}$ for all $k$.
Considering all $k$, we can more simply refer to the components of $\mathcal{W}_s$ as $\widetilde{E}_s$, $\widetilde{F}_s$, and $\widetilde{H}_s$.
With each component of a wavefront represented as a vector of offsets centered around the main diagonal $k = 0$, we can define the highest (or rightmost) and lowest (or leftmost) diagonals in the component as $\widetilde{H}^{hi}_s$ and $\widetilde{H}^{lo}_s$.

%Where our graph $G$ is directed and acyclic, we 

%We generalize the notion of a diagonal, which is critical to the efficient formulation of WFA, to respect the set of segments $S$ in the graph.

%Recurrence for WFA:

In WFA, our goal is to derive a global alignment by computing the minimum $s$ such that any of the furthest-reaching points in $\mathcal{W}_s$ reaches $(n, m)$.
Equation \ref{eq:wfa} shows the recurrence relations defined in WFA, which reformulate the same scoring model in SWG to consider furthest-reaching points in diagonals.

\begin{equation}
\begin{array}{lll}
\widetilde{E}_{s, k}=\max \left\{\begin{array}{ll}
\widetilde{H}_{s-o-e, k-1} & \text { (open gap in $Q$) } \\
\widetilde{E}_{s-e, k-1} & \text { (extend gap in $Q$) }
\end{array}\right\}+1 \\
\widetilde{F}_{s, k}=\max \left\{\begin{array}{ll}
\widetilde{H}_{s-o-e, k+1} & \text { (open gap in $T$) } \\
\widetilde{F}_{s-e, k+1} & \text { (extend gap in $T$) }
\end{array}\right\} \\
\widetilde{H}_{s, k}=\max \left\{\begin{array}{ll}
\widetilde{H}_{s-x, k}+1 & \text { (substitution) } \\
\widetilde{E}_{s, k} & \text { (gap in $Q$) } \\
\widetilde{F}_{s, k} & \text { (gap in $T$) }
\end{array}\right\}
\end{array}
\label{eq:wfa}
\end{equation}

WFA thus computes the furthest-reaching points of $\mathcal{W}_s$ using only $\mathcal{W}_{s - o}$, $\mathcal{W}_{s - e}$, and $\mathcal{W}_{s - x}$.
Note that, with $a=0$, considering matches requires us to extend all previously computed points by as far as possible by following matching characters along the diagonal.
This extension of the wavefront in the case of matches is essential to the good memory and time characteristics of WFA\@.

\section{Wavefront partial order alignment (WFPOA)}

We consider the generalization of WFA to the case where our target is the directed acyclic graph $G$, which we term WFPOA\@.
In general, this extension is similar to that applied to transform SWG into SWG-POA\@.
But, the definition of diagonals in the alignment matrix is no longer stable across the full score matricies $H$, $E$, and $F$.
As such, we must define wavefronts in terms of the diagonals of each segment $c \in C$.

For a score $s$ and segment diagonal $k = j - i$, the furthest-reaching point $\mathcal{R}_{s,k,c}$ indicates the cell in $H$ that is the furthest from the beginning of diagonal $k$ in segment $c$ with score $s$.
$\widetilde{E}_{s, k, c}$, $\widetilde{F}_{s, k, c}$, and $\widetilde{H}_{s, k, c}$ store the offset in the diagonal to furthest-reaching point $\mathcal{R}_{s,k,c}$ in each of the SWG-POA matrices.
For a given score $s$, the $s$-wavefront $\mathcal{W}_s$ is the set of offsets $\widetilde{E}_{s, k, c}$, $\widetilde{F}_{s, k, c}$, and $\widetilde{H}_{s, k, c}$ for all $k$ on all $c$.
Considering all $k$ on all $c$, we can more simply refer to the components of $\mathcal{W}_s$ as $\widetilde{E}_s$, $\widetilde{F}_s$, and $\widetilde{H}_s$.


In WFPOA, our goal is to derive a global alignment by computing the minimum $s$ such that any of the furthest-reaching points in $\mathcal{W}_s$ reaches the end of any of the tails of the graph $V_{t} = \{ v : \delta_v^{out} = \emptyset$ $(n', v) \}$, with $n' < n$, or any of the set of points $\{ \forall_{v\in V} (n, v) \}$.
Equation \ref{eq:wfpoa} shows the recurrence relations defined in WFPOA, which reformulate the same scoring model in SWG-POA to consider furthest-reaching points in diagonals on specific segments.
We depend on function $\phi(k,c,c^\prec)$ which computes the diagonal on the inbound segment $c^\prec$ corresponding to the diagonal $k$ on $c$.

\begin{equation}
\begin{array}{lll}
\widetilde{E}_{s, k, c}=\max_{\forall c^\prec \in \delta_c^{in}}  \left\{\begin{array}{ll}
\widetilde{H}_{s-o-e, \phi(k-1,c,c^\prec), c^\prec} & \text { (open gap in $Q$) } \\
\widetilde{E}_{s-e, \phi(k-1,c,c^\prec), c^\prec} & \text { (extend gap in $Q$) }
\end{array}\right\}+1 \\
\widetilde{F}_{s, k, c}=\max \left\{\begin{array}{ll}
\widetilde{H}_{s-o-e, k+1, c} & \text { (open gap in $T$) } \\
\widetilde{F}_{s-e, k+1, c} & \text { (extend gap in $T$) }
\end{array}\right\} \\
\widetilde{H}_{s, k, c}=\max \left\{\begin{array}{ll}
\max_{\forall c^\prec \in \delta_c^{in}}\{\widetilde{H}_{s-x, \phi(k,c,c^\prec), c^\prec}+1 \} & \text { (substitution) } \\
\widetilde{E}_{s, k, c} & \text { (gap in $Q$) } \\
\widetilde{F}_{s, k, c} & \text { (gap in $T$) }
\end{array}\right\}
\end{array}
\label{eq:wfpoa}
\end{equation}

We also need to extend the concept of distance along a diagonal to record the minimum and maximum possible diagonal traversal distances for each of our furthest-reaching points.

\begin{comment} % notes

Relation of the successive wavefront high and low bounds.

\begin{equation}\begin{array}{l}
\widetilde{H}_{s}^{h i}=\widetilde{E}_{s}^{h i}=\widetilde{F}_{s}^{h i}=\max \left\{\widetilde{H}_{s-x}^{h i}, \widetilde{H}_{s-o-e}^{h i}, \widetilde{E}_{s-e}^{h i}, \widetilde{F}_{s-e}^{h i}\right\}+1 \\
\widetilde{H}_{s}^{l o}=\widetilde{E}_{s}^{l o}=\widetilde{F}_{s}^{l o}=\max \left\{\widetilde{H}_{s-x}^{l o}, \widetilde{H}_{s-o-e}^{l o}, \widetilde{E}_{s-e}^{l o}, \widetilde{F}_{s-e}^{l o}\right\}-1
\end{array}\end{equation}



%\bigbreak
%Recurrence for WFA --- TODO: TO ADAPT FOR WFA



Recurrence for SWG:

\begin{equation}
  \left\{\begin{array}{l}
  I_{v, h}=\min \left\{M_{v, h-1}+o+e, I_{v, h-1}+e\right\} \\
  D_{v, h}=\min \left\{M_{v-1, h}+o+e, D_{v-1, h}+e\right\} \\
  M_{v, h}=\min \left\{I_{v, h}, D_{v, h}, M_{v-1, h-1}+s\left(q_{v-1}, t_{h-1}\right)\right\}
  \end{array}\right.
\end{equation}

Recurrence for WFA:

\begin{equation}
\begin{array}{lll}
\widetilde{I}_{s, k}=\max \left\{\begin{array}{ll}
\widetilde{M}_{s-o-e, k-1} & \text { (Open insertion) } \\
\widetilde{I}_{s-e, k-1} & \text { (Extend insertion) }
\end{array}\right\}+1 \\
\widetilde{D}_{s, k}=\max \left\{\begin{array}{ll}
\widetilde{M}_{s-o-e, k+1} & \text { (Open deletion) } \\
\widetilde{D}_{s-e, k+1} & \text { (Extend deletion) }
\end{array}\right\} \\
\widetilde{M}_{s, k}=\max \left\{\begin{array}{ll}
\widetilde{M}_{s-x, k}+1 & \text { (Substitution) } \\
\widetilde{I}_{s, k} & \text { (Insertion) } \\
\widetilde{D}_{s, k} & \text { (Deletion) }
\end{array}\right\}
\end{array}
\end{equation}

Relation of the successive wavefront high and low bounds.

\begin{equation}\begin{array}{l}
\widetilde{M}_{s}^{h i}=\widetilde{I}_{s}^{h i}=\widetilde{D}_{s}^{h i}=\max \left\{\widetilde{M}_{s-x}^{h i}, \widetilde{M}_{s-o-e}^{h i}, \widetilde{I}_{s-e}^{h i}, \widetilde{D}_{s-e}^{h i}\right\}+1 \\
\widetilde{M}_{s}^{l o}=\widetilde{I}_{s}^{l o}=\widetilde{D}_{s}^{l o}=\max \left\{\widetilde{M}_{s-x}^{l o}, \widetilde{M}_{s-o-e}^{l o}, \widetilde{I}_{s-e}^{l o}, \widetilde{D}_{s-e}^{l o}\right\}-1
\end{array}\end{equation}

\begin{equation}
\left\{\begin{array}{l}
C_{i,v} = \\
C_{i-1,u} + \bigtriangleup_{i,v} $, for$ u \in \delta_v^in \\
C_{i,u} + 1 $, for$ u \in \delta_v^in \\;
C_{i-1,v} + 1
\end{array}\right.
\end{equation}

with the boundary condition $C_{0,v} = \bigtriangleup_{0,v}$ for all $v \in V$, where $\bigtriangleup_{i,v}$ is the
mismatch penalty between node character $\sigma(v \in V )$ and sequence character $s_i$, which is 0 for a match and 1
for a mismatch.

\end{comment}


\bibliographystyle{unsrt}
\bibliography{ref}

\end{document}

